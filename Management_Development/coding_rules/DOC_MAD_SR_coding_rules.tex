\documentclass[
fontsize=12pt,					% font size
paper=a4,						% paper format
twoside=false, 					% one page document
listof=totoc, 					% Add table- and graphics list to table of contents
titlepage, 						% use titlepage environment instead of \maketitle
headsepline, 					% horizontal line under column title
DIV=12,							% Type area setting
cleardoublepage=empty,			% empty page at chapter change
parskip,						% more space after paragraphs
usenglisch
]{scrbook}

\usepackage[setspace=false]{scrhack}
% enables ä,ö,ü
\usepackage[utf8]{inputenc} 	
% All characters in T1-encoding
\usepackage[T1]{fontenc} 		
\usepackage{babel} 
% no indent at new paragraphs
\setlength{\parindent}{0ex} 	
\usepackage[onehalfspacing]{setspace}
% Include PDFs
\usepackage{pdfpages}
% hide the links
\usepackage[hidelinks]{hyperref}
% enable listings
\usepackage{listings}

\usepackage{color}
% enables changes in the numbering with the enumerate environment
\usepackage{enumitem}

\definecolor{mygreen}{rgb}{0,0.6,0}
\definecolor{mygray}{rgb}{0.5,0.5,0.5}
\definecolor{mymauve}{rgb}{0.58,0,0.82}
\lstset{ %
	backgroundcolor=\color{gray},    % choose the background color; you must add \usepackage{color} or \usepackage{xcolor}; should come as last argument
	basicstyle=\footnotesize,        % the size of the fonts that are used for the code
	breakatwhitespace=false,         % sets if automatic breaks should only happen at whitespace
	breaklines=true,                 % sets automatic line breaking
	captionpos=t,                    % sets the caption-position to (b) bottom or (t) top
	commentstyle=\color{mygreen},    % comment style
	deletekeywords={...},            % if you want to delete keywords from the given language
	escapeinside={\%*}{*)},          % if you want to add LaTeX within your code
	escapeinside={(*@}{@*)},
	extendedchars=true,              % lets you use non-ASCII characters; for 8-bits encodings only, does not work with UTF-8
	frame=none,	                   	 % "single" adds a frame around the code; "none"
	keepspaces=true,                 % keeps spaces in text, useful for keeping indentation of code (possibly needs columns=flexible)
	keywordstyle=\color{blue},       % keyword style
	language=C++,                    % the language of the code
	morekeywords={uint8_t,
                  uint16_t,
                  uint32_t, 
                  uint64_t,
                  int8_t,
                  int16_t,
                  int32_t,
                  int64_t,
                  bool},             % if you want to add more keywords to the set
	numbers=left,                    % where to put the line-numbers; possible values are (none, left, right)
	numbersep=5pt,                   % how far the line-numbers are from the code
	numberstyle=\tiny\color{mygray}, % the style that is used for the line-numbers
	rulecolor=\color{black},         % if not set, the frame-color may be changed on line-breaks within not-black text (e.g. comments (green here))
	showspaces=false,                % show spaces everywhere adding particular underscores; it overrides 'showstringspaces'
	showstringspaces=false,          % underline spaces within strings only
	showtabs=false,                  % show tabs within strings adding particular underscores
	stepnumber=1,                    % the step between two line-numbers. If it's 1, each line will be numbered
	stringstyle=\color{mymauve},     % string literal style
	tabsize=2,	                     % sets default tabsize to 2 spaces
	title=\lstname                   % show the filename of files included with \lstinputlisting; also try caption instead of title
}

\begin{document}

\begin{itemize}
    \item Code and comments must be written in English.
    \item Code must be written in C/C++.
    \item Every function must have a doxygen comment above.
    % doxygen rules
    \item Comments must be written above the code descripted.
    \item Pre-defined values must be defined using #define.
    \item include guards must end with six random letters or numbers.
    \item naming
    \begin{itemize}
        \item spaces are written as underscores _.
        \item functions
        \begin{itemize}
            \item First letter must be written in Capital.
        \end{itemize}
        \item variables
        \begin{enumerate}
            \item Capital letters must not be used.
            \item int variables must start with i_
            \item long variables must start with l_
            \item long long variables must start with ll_
            \item uint8_t variables must start with u8_
            \item uint16_t variables must start with u16_
            \item uint32_t variables must start with u32_
            \item uint64_t variables must start with u64_
            \item bool variables must start with bool_
            \item float variables must start with fl_
            \item char variables must start with ch_
            \item strings must start with str_
            \item vectors must start with vec_
            \item transfer parameters begin with an underscore _.
        \end{enumerate}
        \item classes
        \begin{itemize}
            \item classes must start with cla_
        \end{itemize}
    \end{itemize}
    \item All variables in classes must be protected or private.
    % clang-format rules.
\end{itemize}

% ENABLE IF FIGURES ARE IN THE DOCUMENT
%\listoffigures
%\clearpage

% ENABLE IF TABLES ARE IN THE DOCUMENT
%\listoftables
%\clearpage
\end{document}